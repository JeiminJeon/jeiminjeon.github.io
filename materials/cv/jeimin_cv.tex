\documentclass[letterpaper,11pt]{article}

\usepackage{latexsym}
\usepackage[empty]{fullpage}
\usepackage{titlesec}
\usepackage{marvosym}
\usepackage[usenames,dvipsnames]{color}
\usepackage{verbatim}
\usepackage{enumitem}
\usepackage[pdftex,colorlinks=true,linkcolor=blue,urlcolor=blue,citecolor=blue]{hyperref}
\usepackage{fancyhdr}
\usepackage{xcolor}
\usepackage{pbox}


\pagestyle{fancy}
\fancyhf{} % clear all header and footer fields
\fancyfoot{}
%\rhead{~~~~~~~~\thepage}
\renewcommand{\headrulewidth}{0pt}
\renewcommand{\footrulewidth}{0pt}

% Adjust margins
\addtolength{\oddsidemargin}{-0.375in}
\addtolength{\evensidemargin}{-0.375in}
\addtolength{\textwidth}{1in}
\addtolength{\topmargin}{-.5in}
\addtolength{\textheight}{1.0in}

\urlstyle{same}

\raggedbottom
\raggedright
\setlength{\tabcolsep}{0in}

% Sections formatting
\titleformat{\section}{
  \vspace{-4pt}\scshape\raggedright\large
}{}{0em}{}[\color{black}\titlerule \vspace{-5pt}]

%-------------------------
% Custom commands
\newcommand{\resumeItem}[2]{
  \item\small{
    \textbf{#1}{: #2 \vspace{-2pt}}
  }
}


\newcommand{\papername}[1]{``#1''}

\newcommand{\CVPR}{\textit{IEEE Computer Vision and Pattern Recognition}~(\textbf{CVPR})}
\newcommand{\ECCV}{\textit{European Conference on Computer Vision}~(\textbf{ECCV})}
\newcommand{\ICCV}{\textit{IEEE International Conference on Computer Vision}~(\textbf{ICCV})}
\newcommand{\NIPS}{\textit{Conference on Neural Information Processing Systems}~(\textbf{NeurIPS})}
\newcommand{\AAAI}{\textit{AAAI Conference on Artificial Intelligence}~(\textbf{AAAI})}
\newcommand{\PR}{\textit{Pattern Recognition}~(\textbf{PR})}


\newcommand{\resumeSubheading}[5]{
  \vspace{3pt}
  \small
  \begin{tabular*}{0.97\textwidth}{l l@{\extracolsep{\fill}}r}
  	~~~~&\textbf{#1} & #2 \\
  	~~~~&~~#3 & #4 \\
  	~~~~&~~#5 \\
  \end{tabular*}
  \vspace{3pt}
}

\newcommand{\projectlist}[4]{
  \vspace{3pt}
  \small
  \begin{tabular*}{0.97\textwidth}{l@{\extracolsep{\fill}}r}
  	~~\textbf{#1} & #2 \\
  	~~~~#3 & \\
    ~~~~#4 \\
  \end{tabular*}
  \vspace{3pt}
}

\newcommand{\worklist}[5]{
  \vspace{3pt}
  \small
  \begin{tabular*}{0.92\textwidth}{l@{\extracolsep{\fill}}r}
  	~~\textbf{#1}, #2, #3 & #4 \\
  	~~~~#5 \\
  \end{tabular*}
  \vspace{3pt}
}


\begin{document}
% \footnotesize Last updated: \today 
% \vspace{0.5cm}
% \footnotesize Last modified: \today

% \noindent\makebox[\linewidth]{\rule{\textwidth}{0.4pt}}
% hline

\vspace{0.5cm}

\begin{tabular*}{\textwidth}{l@{\extracolsep{\fill}}r}
  \textbf{\Large Jeimin Jeon}  & jeimin@yonsei.ac.kr\\
 \small  \color{darkgray}{PhD student @ Computer Vision Lab, Yonsei University} & \href{https://jeiminjeon.github.io/}{https://jeiminjeon.github.io} \\
\end{tabular*}





%-----------Research Interests-----------------
\section{Research Interests}
~~~~\textbf{Machine Learning and Computer Vision}\\
\vspace{-6pt}
\begin{itemize}
 \item[-] \small{Efficient Model (\emph{e.g.}, NAS, Quantization, Pruning)} \vspace{-4pt}
 \item[-] \small{Large Vision-Language Model} \vspace{-4pt}
 \item[-] \small{Image Generation} \vspace{-4pt}
\end{itemize}


%-----------Education-----------------
\section{Education}
\resumeSubheading
  {Yonsei University}{(Mar. 2022 -- Present)}
  {\textbf{Ph.D.} in Electrical and Electronic Engineering}{}
  {\textit{Advisor}: Prof. Bumsub Ham}

\resumeSubheading
  {Yonsei University}{(Mar. 2016 -- Feb. 2022)}
  {\textbf{B.S.} in Electrical and Electronic Engineering}{}
  {GPA: 4.01/4.3, \textit{Graduated magna cum laude (Top 3\%)}}


%%-----------Publications-----------------
\section{Publications}
~~~~$*$: equal contribution
\begin{enumerate}
  \item \small Junghyup Lee*, \underline{\textbf{Jeimin Jeon}}*, Dohyung Kim, and Bumsub Ham,~\papername{Scheduling Weight Transitions for Quantization-Aware Training},~\ICCV, 2025.
  \item	\small \underline{\textbf{Jeimin Jeon}}, Youngmin Oh, Junghyup Lee, Donghyeon Baek, Dohyung Kim, Chanho Eom, and Bumsub Ham,~\papername{Subnet-Aware Dynamic Supernet Training for Neural Architecture Search},~\CVPR, 2025.
  \item	\small Dohyung Kim, Junghyup Lee, \underline{\textbf{Jeimin Jeon}}, Jaehyeon Moon, and Bumsub Ham,~\papername{Toward INT4 Fixed-Point Training via Exploring Quantization Error for Gradients},~\ECCV, 2024.
  
  \vspace{0.2cm}
  
  
  \textbf{\hspace{-0.8cm} Under Review}
  \item \textbf{First Author}. \textit{Transformer Architecture Search with Mixture-of-LoRA Experts}, under review.
  \item \textbf{Co-Author}. \textit{AccuQuant: Simulating Multiple Denoising Steps for Quantizing Diffusion Models}, under review.

\end{enumerate}





%%-----------Projects-----------------
\section{Projects}
\projectlist
  {\pbox{25cm}{Edge artificial intelligence semiconductor IP development}}{(Aug. 2023 -- Present)}
  {Korea Technology \& Information Promotion Agency for SMEs (TIPA)}
  {\begin{tabular*}{0.5\textwidth}{l@{\extracolsep{\fill}}}
    ~~- Developed quantization and pruning algorithms for In-Memory Computing (IMC) chips. \\
  	~~- Collaborated with hardware teams for efficient HW-SW co-design. \\
  	~~- Built deep learning models for circuit performance prediction and optimization.\\
  \end{tabular*}}
  
\projectlist
  {\pbox{25cm}{Development of Fundamental Technology and Integrated Solution for \\Next-Generation Automatic Artificial Intelligence System}}{(Apr. 2022 -- Jul. 2023)}
  {Institute for Information \& Communications Technology Promotion (IITP)}
  {\begin{tabular*}{0.5\textwidth}{l@{\extracolsep{\fill}}}
    ~~- Developed neural architecture search (NAS) algorithms for CNNs, ViTs, and quantized models. \\
  	~~- Designed Automatic Loss Function Search algorithms for adaptive optimization. \\
  	~~- Implemented low-bit training techniques for efficient deep learning model training.\\
  \end{tabular*}}

  % \clearpage
\section{Patents}
\subsection*{International}
\begin{itemize}
  \item[-]	\small Dynamic Supernet Learning Apparatus and Method for Neural Architecture Search \\
  \footnotesize  \textcolor{gray}{US18799660, Aug. 2024 (Application)}
\end{itemize}

\subsection*{Domestic}
\begin{itemize}
  \item[-]	\small Apparatus and Method for Quantizing Tokens of Vision Transformers \\
  \footnotesize  \textcolor{gray}{10-2024-0137421, Oct. 2024 (Application)}
  \item[-]	\small Dynamic Supernet Learning Apparatus and Method for Neural Architecture Search \\
  \footnotesize  \textcolor{gray}{10-2024-0100942, Jul. 2024 (Application)}
  \item[-]	\small Quantization Apparatus and Method for Artificial Neural Network \\
  \footnotesize  \textcolor{gray}{10-2023-0116857, Sep. 2023 (Application)}
  \item[-] \small Quantization-Aware Training Apparatus and Method \\
  \footnotesize  \textcolor{gray}{10-2023-0049837, Apr. 2023 (Application)}
\end{itemize}

%%-----------Experiences-----------------
\section{Experiences}
\begin{itemize}
  \item[-] \textbf{Peer-review Activity} 
  \\
  \vspace{5pt}
  \begin{tabular*}{0.25\textwidth}{l@{\extracolsep{\fill}}l}
  	~~CVPR & 2024,2025 \\
    ~~NeurIPS & 2025 \\
  	~~ECCV & 2024 \\
  \end{tabular*}

%  \item[-] \textbf{Talk}
%  \\
%  \vspace{5pt}
%  \worklist
%  	{2022 Korean AI Association \& Naver Fall Joint Conference}{Seoul}{South Korea}{(Nov.2022)}
%  	{ALIFE: Adaptive Logit Regularizer and Feature Replay for Incremental Semantic Segmentation}

  \item[-] \textbf{Teaching Assistant}
  \\
  \vspace{2pt}
  \begin{tabular*}{0.5\textwidth}{l@{\extracolsep{\fill}}}
  	~~Deep Learning Lab (EEE4423): 2022-1, 2024-1, 2025-1 \\
  	~~Digital Image Processing (EEE5320): 2023-2 \\
  	~~Electrical and Electronic Engineering 101 (EEE2113): 2023-1 \\
  	~~SW Programming (YCS-1002): 2021-1
  \end{tabular*}
\end{itemize}

\end{document}
